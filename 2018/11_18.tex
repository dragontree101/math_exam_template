\documentclass[12pt,landscape,UTF8,onecolumn]{ctexart}
\usepackage[utf8]{inputenc}

\usepackage[paperwidth=21cm,paperheight=29.7cm,
top=2cm,bottom=2cm,right=2cm,left=2cm]{geometry}

\usepackage{amsmath}
\usepackage{amsfonts}
\usepackage{amssymb}

\usepackage{makeidx}

\usepackage{graphicx}
\usepackage{setspace}
\usepackage{tabu}
\usepackage{paralist}
\usepackage{lastpage}
\usepackage{enumerate} 

\usepackage{pgf,tikz,pgfplots}
\usepackage{mathrsfs}

\usepackage{fancyhdr}
\renewcommand{\headrulewidth}{0pt}
\pagestyle{fancy}
\fancyfoot[CO]{第~\thepage~页~~共~\pageref{LastPage}~页}

\newcommand{\putzdx}{
		\hspace{-1.7cm}\parbox{1cm}{\vspace{-1.5cm}
			\rotatebox[origin=c]{90}{
				\usebox{\zdxa}
		}}
}
\newcommand{\putzdxx}{
	\hspace{0.3cm}\parbox{1cm}{\vspace{-1.5cm}
		\rotatebox[origin=c]{-90}{
			\usebox{\zdxb}
	}}
}


\usepackage{ifthen}

%选择题选项命令 \xx \xxiii \xxv \xxvi
\newlength{\la}
\newlength{\lb}
\newlength{\lc}
\newlength{\ld}
\newlength{\lee}
\newlength{\lf}
\newlength{\lhalf}
\newlength{\lquarter}
\newlength{\lmax}
\newcommand{\xx}[4]{\\[.5pt]%  
	\settowidth{\la}{A、#1~~~}  
	\settowidth{\lb}{B、#2~~~}
	\settowidth{\lc}{C、#3~~~}  
	\settowidth{\ld}{D、#4~~~}  
	\ifthenelse{\lengthtest{\la > \lb}}
	{\setlength{\lmax}{\la}}{\setlength{\lmax}{\lb}}  
		\ifthenelse{\lengthtest{\lmax < \lc}}  {\setlength{\lmax}{\lc}}  {}  \ifthenelse{\lengthtest{\lmax < \ld}}  {\setlength{\lmax}{\ld}}  {} 
	    \setlength{\lhalf}{0.5\linewidth} 
	    \setlength{\lquarter}{0.25\linewidth}
	    \ifthenelse{\lengthtest{\lmax > \lhalf}}  
	    {\noindent{}A、#1 \\ B、#2 \\ C、#3 \\ D、#4 }  {  \ifthenelse{\lengthtest{\lmax > \lquarter}}  
	    	{\noindent
	    	 \makebox[\lhalf][l]{A、#1~~~}% 
	    	 \makebox[\lhalf][l]{B、#2~~~}\\%
	    	 \makebox[\lhalf][l]{C、#3~~~}%
	    	 \makebox[\lhalf][l]{D、#4~~~}}% 
    		 {\noindent\makebox[\lquarter][l]{A、#1~~~}% 
    		 \makebox[\lquarter][l]{B、#2~~~}%     
    		 \makebox[\lquarter][l]{C、#3~~~}%      
    		 \makebox[\lquarter][l]{D、#4~~~}}
    	 }}

%填空题画线  \tk
\newcommand{\tk}[1][1.5]{\;\underline{\mbox{\hspace{#1 cm}}}\,}

\newif\ifprint
\printfalse
\usepackage{ulem}
\newcommand{\tkf}[2][3cm]{\uline{\makebox[#1][c]{%
			\ifprint
			\phantom	{#2#2}%
			\else
			#2%
			\fi}}}

\begin{document}
  \noindent	
  \begin{spacing}{1.5}
    \begin{center}
      \zihao{3} \heiti 湘一芙蓉中学~2018~年下学期周考

      初一~~数学~~试卷

      \zihao{4} 时量\underline{~ 45分钟 ~}总分\underline{~ 100分 ~} 命题人\underline{~ 孙晓萍 ~} 审稿\underline{~ 孙晓萍 ~}
    
    \end{center}
  \end{spacing}
  \vspace{-10pt}\zihao{5} 
  \begin{spacing}{1.3}
    \begin{enumerate} [1、]
      \item[\heiti 一、] {\heiti 选择题(每空6分,共30分)}

      \item 下列选项中最大值的是( )
      \xx{$2^{53}$}{$3^{33}$}{$5^{23}$}{$7^{19}$}

      \item 一位登山者早上6点从山脚出发,下午四点到达山顶,第二天早上6点从山顶出发,按原路返回,下午四点回到山脚,以下说法正确的是( )
      \xx{一定有一个时间点,这位登山者第一天和第二天处在山的同一个位置}{有可能存在一个时间点,这位登山者第一天和第二天处在山的同一个位置}{一定没有一个时间点,这位登山者第一天和第二天处在山的同一个位置}{以上都不正确}

      \item 存在以下等式$\dfrac x{y+z} + \dfrac y{x+z} + \dfrac z{x+y} = 4$那么下面说法正确的是( )
      \xx{$~x~$,$~y~$,$~z~$没有整数解}{$~x~$,$~y~$,$~z~$只有正整数解}{$~x~$,$~y~$,$~z~$有多个整数解}{$~x~$,$~y~$,$~z~$有且只有一个整数解}

      \item 一种商品先涨价10\%,后又降价10\%,现在的商品价格与原来相比( )
      \xx{升高了}{降低了}{没有变化}{根据商品价格而定}

      \item 20人乘飞机从甲市前往乙市,总费用为27000元。每张机票的全价票单价为2000元,除全价票之外,该班飞机还有九折票和五折票两种选择。每位旅客的机票总费用除机票价格之外,还包括170元的税费。 下面选项正确的是( )
      \xx{两者一样多}{买九折票的多1人}{买全价票的多2人}{买九折票的多4人}

      \item[\heiti 二、] { \heiti 填空题(每空6分,共24分) }

      \item 111支球队两两捉对进行淘汰赛,没有对手则轮空,问最后决出冠军,则需要打\tk{}场比赛

      \item 有1000克蘑菇,它们的含水量是99\%,稍经晾晒,含水量下降到98\%,晾晒后的蘑菇重\tk{}千克

      \item 通过四个数字$6,9,9,10$计算24(只能使用加、减、乘、除),表达式为\tk{}

      \item 记符号$\prod_{i=1}^4$的含义是$1*2*3*4$,$\prod_{i=1}^4$的值为$24$,那么$\prod_{i=-100}^{100}$的值为\tk{}

      \item[\heiti 三、] {\heiti 解答题(共46分)}

      \item (8分)佩奇今年6岁,每年只增长半岁,金龙刚刚出生,每年增长一岁问,多少年之后,金龙的年纪是佩奇的二倍?

      \item (12分)晚上过桥需要用手电筒,小明一家要过桥,小明要用1分钟, 弟弟要用3分钟, 爸爸要用6分钟,妈妈要用8分钟,爷爷要用12分钟,一次最多只能过两个人,过桥后要从已过桥的人中选一个人将手电筒拿回去给其他人用,所有人都要过桥,最少要用多少时间? 具体的过桥的方法是什么样子的?

      \item (12分)甲、乙两人分别从相距 100 米的 A 、B 两地出发,相向而行,其中甲的速度是 2 米每秒,乙的速度是 3 米每秒。一只狗从 A 地出发,先以 6 米每秒的速度奔向乙,碰到乙后再掉头冲向甲,碰到甲之后再跑向乙,如此反复,直到甲、乙两人相遇。问在此过程中狗一共跑了多少米? 

      \item (14分)完成下列数独(每行、每列、每个九宫格都由$1-9$中9个数字组成)
        \begin{sudoku}
          | |2| | |3| |9| |7|.
          | |1| | | | | | | |.
          |4| |7| | | |2| |8|.
          | | |5|2| | | |9| |.
          | | | |1|8| |7| | |.
          | |4| | | |3| | | |.
          | | | | |6| | |7|1|.
          | |7| | | | | | | |.
          |9| |3| |2| |6| |5|.
        \end{sudoku}

    \end{enumerate} 
  \end{spacing}
\end{document}